\section{Fixed-parameter parallelizability}
\label{sec:pcompletefpp}

% % Foreword %
%
% %% Context (anyone - why now?) %%
%
% What is the current situation, and why is the need so important?
%
Classical computational complexity has a well-developed theory of parallel versus sequential computation.
Those computational problems that are $\P$-complete are inherently sequential, whereas those in the class $\NC$ admit highly parallel algorithms.
Even though they can be solved in polynomial time, adding more processors does not provide any significant reduction in the time required to find a solution for $\P$-complete problems.
% %% Need (readers - why you?) %%
%
% Why is this relevant to the reader, and why does something need to be done?
% (Also reference relevant existing work.)
%
Can the lack of parallelizability for $\P$-complete problems depend on the parameterization of the problem, thereby isolating some of the ``sequential-ness'' of the problem in the value of the parameter?
In other words, can parameterization of problems that have traditionally been considered inherently sequential afford us a new avenue for parallelization?

%
% %% Task (author - why me?) %%
%
% What was undertaken to address the need?
%
Specifically, we would like to determine whether there are problems that are inherently sequential in the classical sense but parallelizable under some parameterization.
%
% %% Object (document - why this document?) %%
%
% What does this document cover?
%
This section combines and adapts the definitions of parallelization from classical complexity theory and the definitions of parameterized complexity theory for highly parallel problems.
%

% % Summary %
%
% %% Findings (author - what?)
%
% What did the work reveal when performing the task?
%
We provide the definition of $\para \NC$, demonstrate the relationship between $\para \NC$ and $\NC$, and prove some sufficient conditions for membership in $\para \NC$.
%
% %% Conclusion (readers - so what?)
%
% What did the findings mean for the audience?
%
These results demonstrate that the idea of parameterized parallelizability is both meaningful and interesting for seemingly inherently sequential problems.
%
% %% Perspective (anyone - what now?)
%
% What should be done next?
Subsequent sections will examine the limits to parameterized parallel computation.

\subsection{Definition of \texorpdfstring{$\para \NC$}{paraNC}}

% % Foreword %
%
% %% Context (anyone - why now?) %%
%
% What is the current situation, and why is the need so important?
%
%
% %% Need (readers - why you?) %%
%
% Why is this relevant to the reader, and why does something need to be done?
% (Also reference relevant existing work.)
%
%
% %% Task (author - why me?) %%
%
% What was undertaken to address the need?
%
%
% %% Object (document - why this document?) %%
%
% What does this document cover?
%
%

% % Summary %
%
% %% Findings (author - what?)
%
% What did the work reveal when performing the task?
%
%
% %% Conclusion (readers - so what?)
%
% What did the findings mean for the audience?
%
%
% %% Perspective (anyone - what now?)
%
% What should be done next?

The $\para$ ``operator'' defined in \autocite{fg03} applies generically to an arbitrary complexity class as follows.
If $\mathcal{C}$ is a class of decision problems, then $\para \mathcal{C}$ is the class of parameterized problems $(Q, \kappa)$ for which there is a decision problem $L \in \mathcal{C}$ and a computable function $f$ such that $x$ in $Q$ if and only if $(x, 1^{f(\kappa(x))}) \in L$.
When $\mathcal{C} = \NC$ in particular, we get the following equivalent definition.

\begin{definition}[$\para \NC^d$]
  Let $d$ be a natural number.
  A parameterized problem $(Q, \kappa)$ is in the class $\para \NC^d$ if there is a circuit-computable function $f$ and a nonuniform family $\{C_{n, k}\}$ of bounded fan-in Boolean circuits such that for each string $x$,
  \begin{itemize}
  \item $x \in Q$ if and only if $C_{n, k}(x) = 1$, where $n = |x|$ and $k = \kappa(x)$,
  \item $\size(C_{n, k}) \leq f(k) n^{O(1)}$,
  \item $\depth(C_{n, k}) \leq f(k) + O(\log^d n)$.
  \end{itemize}
\end{definition}

If the depth of the circuit is instead bounded by $f(k) O(\log^d n)$, the class is denoted $\para \NC^{d \uparrow}$, a superclass of $\para \NC^d$.
If the circuits are of unbounded fan-in, the classes are $\para \AC^d$ and $\para \AC^{d \uparrow}$, respectively.
The classes $\para \AC^{d \uparrow}$ were first defined in \autocite{bst15}.

A subtle point is that the value of the parameter $\kappa(x)$ must be non-constant but also independent of the size of the instance $x$ for the parameterized problem to be interesting.
First, if $\kappa(x)$ were bounded above by a constant for each $x$, then the parameter would be irrelevant and the problem would simply be in the standard complexity class $\NC^d$.
Thus depth $O(\log^d n)$ and depth $f(k) \log^d n$ are different.
On the other hand, if $\kappa(x)$ were bounded from below by a nondecreasing, unbounded function of $|x|$, then the problem would be trivially in $\para \AC^{0 \uparrow}$ by the technique of \autocite[Proposition~1.7]{fg06}.
Thus a formula like $\log^2 (kn)$, which may appear in the analysis of certain simulations of parameterized complexity classes (see \autoref{lem:pfsatmem}, for example), becomes
\[
\log^2 (kn) = (\log k + \log n)^2 = \log^2 k + \log k \log n + \log^2 n \leq \log^2 k + 2 \log^2 n,
\]
and thus $\log^2 (kn) = f(k) + O(\log^2 n)$ for some computable function $f$.

\subsection{Example problem in \texorpdfstring{$\para \NC$}{paraNC}}
\label{sec:parancexample}

% % Foreword %
%
% %% Context (anyone - why now?) %%
%
% What is the current situation, and why is the need so important?
%
Useful complexity classes are nonempty and have interesting natural problems,
%
% %% Need (readers - why you?) %%
%
% Why is this relevant to the reader, and why does something need to be done?
% (Also reference relevant existing work.)
%
so in an effort to show that $\para \NC$ has parameterized problems whose underlying decision problems are $\P$-complete,
%
% %% Task (author - why me?) %%
%
% What was undertaken to address the need?
%
we consider parameterizations of the canonical $\P$-complete problem, the circuit evaluation problem.
%
% %% Object (document - why this document?) %%
%
% What does this document cover?
%
%
This section provides a non-degenerate parameterization of the circuit evaluation problem that makes it parallelizable.

% % Summary %
%
% %% Findings (author - what?)
%
% What did the work reveal when performing the task?
%
We provide a $\para \NC$ problem based on a $\P$-complete problem with a degenerate parameterization (\autoref{thm:degen}), one with a nondegenerate parameterization (\autoref{thm:cvpdepth}), and one based on an optimization problem (\autoref{thm:onesdepth}).
(Contrast these results with the parameterized vertex cover problem, which is in $\para \AC^0$ \autocite{bst15}, but whose underlying decision problem is $\NP$-complete.)
%
% %% Conclusion (readers - so what?)
%
% What did the findings mean for the audience?
%
Thus $\para \NC$ does indeed contain interesting problems
%
% %% Perspective (anyone - what now?)
%
% What should be done next?
and we can use these as the starting point for studying the limits of parameterized parallelization.

We start by choosing $Q$ to be a $\P$-complete problem and $\kappa$ to be the ``degenerate'' parameterization function $\kappa(x) = |x|$.
The \emph{circuit evaluation problem} is the problem of deciding whether, given a Boolean circuit and an input to that circuit, the output of the circuit is 1.

%% \begin{definition}[$p \dash s \dash \textsc{Circuit Evaluation}$]
%%   \mbox{} \\
%%   \begin{tabular}{r p{9.2cm}}
%%     \textbf{Instance:} & Boolean circuit $C$ on $n$ inputs with size $m$ and depth $d$, binary string $x$ of length $n$. \\
%%     \textbf{Parameter:} & $m$. \\
%%     \textbf{Question:} & Does $C(x) = 1$?
%%   \end{tabular}
%% \end{definition}

\begin{theorem}\label{thm:degen}
  The circuit evaluation problem parameterized by the size of the circuit is in $\para \NC$ and the underlying decision problem is $\P$-complete.
\end{theorem}
\begin{proof}
  The circuit evaluation problem is $\P$-complete by \autocite{ladner75}.
  Since the parameterization is monotonically increasing with the size of the input, the problem is in $\para \NC$ by the technique of \autocite[Proposition~1.7]{fg06}.
\end{proof}

To find a non-degenerate example, we can parameterize the circuit evaluation problem by depth instead of size.

%% \begin{definition}[$p \dash d \dash \textsc{Circuit Evaluation}$]
%%   \mbox{} \\
%%   \begin{tabular}{r p{9.2cm}}
%%     \textbf{Instance:} & Boolean circuit $C$ on $n$ inputs with size $m$ and depth $d$, binary string $x$ of length $n$. \\
%%     \textbf{Parameter:} & $d$. \\
%%     \textbf{Question:} & Does $C(x) = 1$?
%%   \end{tabular}
%% \end{definition}

\begin{theorem}\label{thm:cvpdepth}
  The circuit evaluation problem parameterized by the depth of the circuit is in $\para \AC^{0 \uparrow}$ and the underlying decision problem is $\P$-complete.
\end{theorem}
\begin{proof}
  As stated in the proof of the previous theorem, the circuit evaluation problem is $\P$-complete.
  Evaluating the circuit $C$ of size $m$ and depth $d$ on inputs $x$ can be performed by the depth-universal circuit $U$ of \autocite{ch85}.
  The size of $U$ is $O(m)$ and the depth is $d$, so there is a function $f$ such that the size is bounded by $f(d) m^{O(1)}$ and the depth by $f(d)$.
  Therefore the circuit evaluation problem parameterized by circuit depth is in $\para \AC^{0 \uparrow}$.
\end{proof}

For a problem with a standard parameterization derived from an optimization problem (see \autoref{def:standard} below), consider the ``depth of ones'' problem.
The \emph{depth of ones} problem is the problem of deciding, given a circuit, an input to the circuit, and a positive integer $k$, whether a $1$ appears at depth at least $k$ when evaluating the circuit on the input.
The circuit evaluation problem is a special case of the depth of ones problem if we choose $k$ to be the depth of the circuit $C$.
%%
%% \begin{definition}[$p \dash \textsc{Ones Depth}$]
%%   \mbox{} \\
%%   \begin{tabular}{r p{9.2cm}}
%%     \textbf{Instance:} & Boolean circuit $C$ on $n$ inputs with size $m$ and depth $d$, binary string $x$ of length $n$, positive integer $k$. \\
%%     \textbf{Parameter:} & $k$. \\
%%     \textbf{Question:} & When evaluating $C(x)$, do ones propagate to depth at least $k$?
%%   \end{tabular}
%% \end{definition}
%%
As an optimization problem, the depth of ones problem is inapproximable up to any constant factor by any $\NC$ circuit, unless $\NC = \P$ \autocite{ks88}.
Contrast this with the complexity of the corresponding parameterized problem.

\begin{theorem}\label{thm:onesdepth}
  The depth of ones problem parameterized by the depth parameter $k$ is in $\para \AC^{0 \uparrow}$ and the underlying decision problem is $\P$-complete.
\end{theorem}
\begin{proof}
  Computing the depth of ones in a circuit is $\P$-complete \autocite{ks88} (see also \autocite[Problem~A.1.10]{ghr95}).
  The naïve algorithm for solving this problem is to take the subcircuit consisting of all gates starting from the inputs and extending through layer $k$, evaluating that (multi-output) circuit, then applying a single \textsc{or} gate to decide whether any of the gates at layer $k$ evaluated to one.
  For each gate at layer $k$, use an instance of the depth-universal circuit to evaluate the single-output circuit induced by that gate.
  This yields a circuit of depth $O(k)$ and size $f(k) m^{O(1)}$ for some $f$, where $m$ is the size of the circuit given as input.
  Therefore this problem is in $\para \AC^{0\uparrow}$.
\end{proof}

\subsection{Relationship between \texorpdfstring{$\para \NC$}{paraNC} and \texorpdfstring{$\NC$}{NC}}

% % Foreword %
%
% %% Context (anyone - why now?) %%
%
% What is the current situation, and why is the need so important?
%
How do the parallelizable parameterized problems relate to classical parallelizable computational problems?
%
% %% Need (readers - why you?) %%
%
% Why is this relevant to the reader, and why does something need to be done?
% (Also reference relevant existing work.)
%
In order to determine the conditions under which a parameterized parallel algorithm implies a classical parallel algorithm (and vice versa),
%
% %% Task (author - why me?) %%
%
% What was undertaken to address the need?
%
we consolidate and adapt some results that appear scattered across several parameterized complexity papers and books.
%
% %% Object (document - why this document?) %%
%
% What does this document cover?
%
This section provides a generic technique for constructing a classical parallel algorithm from a parameterized one, and vice versa.

% % Summary %
%
% %% Findings (author - what?)
%
% What did the work reveal when performing the task?
%
Specifically, we show two main lemmas.
\autoref{lem:spreduction} shows how to construct a parameterized parallel algorithm from a parameter-restricted reduction to a highly parallel decision problem.
\autoref{lem:reducetonc} shows how to construct a classical parallel algorithm for from a parameter-restricted reduction to a parameterized parallel problem.
%
% %% Conclusion (readers - so what?)
%
% What did the findings mean for the audience?
%
These give explicit techniques for transforming a parameterized parallel algorithm into a classical parallel algorith and vice versa.
%
% %% Perspective (anyone - what now?)
%
% What should be done next?
They will be used in later sections to provide evidence against the collapse of larger complexity classes to $\para \NC$.

We begin with a lemma that allows us to construct a function that behaves like an upper bound on the inverse of another function.

\begin{lemma}\label{lem:upperinverse}
  For each nondecreasing, unbounded, circuit-computable function $i$, there is a function $f_i$ such that $f_i(i(n)) \geq n$ for each $n \geq f(1)$.
  Furthermore, $f_i$ is nondecreasing, unbounded, and circuit-computable.
  (We call $f_i$ the ``upper inverse'' of $i$.)
\end{lemma}
\begin{proof}
  Define $f_i$ by
  \[
  f_i(k) = \max\{ n_0 \in \mathbb{N} \, | \, \forall n \geq n_0 \colon i(n) \geq k \}.
  \]
  Since $i$ is nondecreasing and unbounded, so is $f_i$.

  To compute $f_i$, we use the fact that $i$ is nondecreasing is unbounded.
  We know that for each $k$ there is a natural number $n_k$ such that for all $n \geq n_k$, we have $i(n) \geq k$.
  Thus the algorithm for computing $f_i$ take $k$ as input and performs a binary search on $i(1), \dotsc, i(n_k)$ to determine the largest $n$ such that $i(n) \geq k$.
  There will be at most $\log n_k$ comparison subcircuits, each requiring a computation of $i$ and a comparison with the integer $k$ (in binary, say), so the overall depth of the circuit computing $f_i$ is $O(\depth(i) \log n \log \log k)$ and the size is $O(\size(i) \log n \log k)$.
\end{proof}

\begin{definition}\label{def:spreduction}
  Suppose $d$ is a natural number, $(Q, \kappa)$ is a parameterized problem, and $Q'$ is a decision problem.
  There is a \emph{small parameter $\NC^d$ many-one reduction} from $(Q, \kappa)$ to $Q'$ if there is a nondecreasing, unbounded, circuit-computable function $i$ and an $\NC^d$ family of circuits $\{R_n\}_{n \in \mathbb{N}}$ such that for each string $x$ of length $n$ with $\kappa(x) \leq i(n)$, we have $x \in Q$ if and only if $R_n(x) \in Q'$.
\end{definition}

%% This lemma is analogous to \autoref{thm:eventually}.
This lemma essentially demonstrates that the closure of $\NC$ under small parameter reductions is a subset of $\para \NC$.
We attempted to show that the closure equals $\para \NC$ but were unable to do so.

\begin{lemma}\label{lem:spreduction}
  Suppose $d$ is a natural number, $(Q, \kappa)$ is a parameterized problem, and $Q'$ is a decision problem.
  If $Q$ is circuit-decidable with uniform size and depth, $Q'$ is in $\NC^d$, and there is a small parameter $\NC^d$ many-one reduction from $(Q, \kappa)$ to $Q'$, then $(Q, \kappa)$ is in $\para \NC^d$.
\end{lemma}
\begin{proof}
  Let $i$ be the function that defines the upper bound on the parameter, below which there is an $\NC^d$ many-one reduction from $Q$ to $Q'$.
  Let $\{R_n\}$ be the $\NC^d$ circuit family computing the reduction.
  The nonuniform family of circuits $\{A_{n, k}\}$ that decides $(Q, \kappa)$ is defined by
  \[
  A_{n, k} =
  \begin{cases}
    C_n^1 & \text{if } i(n) < k \\
    C_{n'}^2 \circ R_{n} & \text{otherwise},
  \end{cases}
  \]
  where $\{C_n^1\}$ is the family of circuits that decides $Q$ with uniform size and depth, $\{C_n^2\}$ is the family of $\NC^d$ circuits that decides $Q'$, and $n'$ is the number of output bits of $R_n$.
  The correctness of $A_{n, k}$ follows from the correctness of the subsequent circuits.
  The circuit family is necessarily nonuniform: the computation of $i(n)$ and $k$ and the comparison of the two decides nonuniformly which circuit to select when defining $A_{n, k}$.

  If $i(n) \geq k$, then the size and depth of the circuit are polynomial and polylogarithmic in $n$, respectively, because the size and depth of $C_{n'}^2$ and $R_n$ are.
  For the case when $i(n) < k$, consider the upper inverse $f_i$ of $i$ guaranteed by \autoref{lem:upperinverse}.
  By construction, $n \leq f_i(i(n)) < f_i(k)$.
  Now
  \begin{align*}
    \size(A_{n, k}) & = \size(C_n^1) = S(n) \leq S(f_i(k)), \\
    \depth(A_{n, k}) & = \depth(C_n^1) = D(n) \leq D(f_i(k)),
  \end{align*}
  where $S$ and $D$ are the (circuit-computable, nondecreasing) size and depth bounds for the circuit family $\{C_n^1\}$.
  Thus in either case, there is a sufficiently large circuit-computable function $f$ such that the size of $A_{n, k}$ is bounded above by $f(k) n^{O(1)}$ and the depth $f(k) + O(\log^d n)$.
\end{proof}

As an aside, let us consider the nonuniformity requirement in this lemma.
All subsequent theorems that require nonuniform circuits are nonuniform because they rely on this lemma, and it is not clear whether this lemma can be made uniform.
Nonuniformity is necessary here, but only a \emph{single bit} of nonuniform advice is required to select the appropriate circuit for $A_{n, k}$, given the length $n$ and the parameterization $k$ of the input.
If the circuit $A_{n, k}$ were implemented with a selector for $i(n) < k$ and both branches as subcircuits, then the overall depth of the circuit would be bounded above by the larger of the depths of the two subcircuits.
For large values of $k$, this could be too great a depth to qualify as $\para \NC$.

On the other hand, if the function $i$ were computable by, for example, a deterministic logarithmic space Turing machine, then we would be able to conclude that $(Q, \kappa)$ is in $\para \L$-uniform $\para \NC$ (assuming we have an appropriate definition for such a class).
In this paper, $i$ will not have that restriction, so we suggest considering parameterized uniformity in future research.

We continue with a corollary of the previous lemma.
The following corollary highlights the special case of the preceding lemma in which the reduction is the identity function.

\begin{corollary}\label{cor:sprself}
  Suppose $(Q, \kappa)$ is a parameterized problem, $d$ is a positive integer, and $i$ is an unbounded, nondecreasing, circuit-computable function.
  Let $i(n) \dash Q$ denote the problem of deciding, given $x$ with $\kappa(x) \leq i(|x|)$, whether $x \in Q$.
  If $i(n) \dash Q$ is in $\NC^d$, then $(Q, \kappa)$ is in $\para \NC^d$.
\end{corollary}
\begin{proof}
  The identity function is a small parameter $\NC^d$ many-one reduction from $(Q, \kappa)$ to $i(n) \dash Q$, thereby proving that $Q$ is in $\para \NC^d$ by the previous lemma.
\end{proof}

This lemma shows that a many-one reduction to a fixed-parameter parallelizable problem can sometimes induce a highly parallel algorithm, if the parameter functions are bounded for the reduced instance.

\begin{lemma}\label{lem:reducetonc}
  Suppose $d$ is a positive integer, $Q$ is a decision problem, and $(Q', \kappa')$ is a parameterized problem.
  Suppose there is an $\NC^d$ many-one reduction from $Q$ to $Q'$, given by the circuit family $\{R_n\}$, and $(Q', \kappa')$ is in $\para \NC^d$ by a circuit family $\{C_{m, k}\}$ of size $f(k) m^{O(1)}$ and depth $f(k) + O(\log^d m)$ on inputs of length $m$.
  If $f(\kappa'(R_n(x))) \leq \min(n^{O(1)}, O(\log^d n))$, then $Q$ is in $\NC^d$.
\end{lemma}
\begin{proof}
  The circuit family that decides $Q$ is $\{A_n\}$, defined by $A_n = C_{m, k} \circ R_n$, where $m$ is the size of the output of $R_n$ and $k = \kappa'(R_n(x))$.
  Since $\size(R_n) = n^{O(1)}$, we have $m = n^{O(1)}$ as well.
  For correctness,
  \[
  x \in Q \iff R_n(x) \in Q' \iff C_{m, k}(R_n(x)) = 1.
  \]
  For size and depth bounds,
  \begin{align*}
    \size(A_n) & = \size(C_{m, k}) + \size(R_n) \\
    & = f(k) m^{O(1)} + n^{O(1)} \\
    & = f(k) n^{O(1)} + n^{O(1)} \\
    & = n^{O(1)} n^{O(1)} + n^{O(1)} \\
    & = n^{O(1)},
  \end{align*}
  and
  \begin{align*}
    \depth(A_n) & = \depth(C_{m, k}) + \depth(R_n) \\
    & = f(k) + O(\log^d m) + O(\log^d m) \\
    & = f(k) + O(\log^d m) \\
    & = f(k) + O(\log^d n) \\
    & = O(\log^d n) + O(\log^d n) \\
    & = O(\log^d n). \qedhere
  \end{align*}
\end{proof}

The following corollary highlights the special case of the preceding lemma in which the decision problem of interest is a ``bounded-parameter'' version of the decision problem underlying the fixed-parameter parallelizable problem; compare this with \autoref{cor:sprself}.
Below, a ``nontrivial'' parameterized problem is one in which $\emptyset \subsetneq Q \subsetneq \{0, 1\}^*$.

\begin{corollary}\label{cor:reducetoself}
  Suppose $(Q, \kappa)$ is a nontrivial parameterized problem, $d$ is a positive integer, and $i$ is an unbounded, nondecreasing, circuit-computable function.
  Let $i(n) \dash Q$ denote the problem of deciding, given $x$ with $\kappa(x) \leq i(|x|)$, whether $x \in Q$.
  If $(Q, \kappa)$ is in $\para \NC^d$ by a circuit family $\{C_{n, k}\}$ of size $f(k) n^{O(1)}$ and depth $f(k) + O(\log^d n)$ on inputs of length $n$ and $f(i(n)) \leq \min(n^{O(1)}, O(\log^d n))$, then $i(n) \dash Q$ is in $\NC^d$.
\end{corollary}
\begin{proof}
  We will show a many-one reduction from the decision problem $i(n) \dash Q$ to the decision problem $Q$ underlying the parameterized problem $(Q, \kappa)$ that satisfies the conditions of the previous lemma.
  The reduction $\{R_n\}$ is defined as follows.
  \[
  R_n(x) =
  \begin{cases}
    x & \text{if } \kappa(x) \leq i(n), \\
    \bot & \text{otherwise},
  \end{cases}
  \]
  where $\bot$ is an arbitrary string not in $Q$ (which must exist because the problem is nontrivial by hypothesis).
  As long as $\kappa$ is computable by an $\NC^d$ circuit family, then so is $R_n$.
  (The computation of $i(n)$ is captured by the nonuniformity of the circuit family, so it does not affect the size or depth required by the circuit computing $R_n$.)

  The reduction $R_n$ is a correct many-one reduction.
  If $x \in i(n) \dash Q$, then $\kappa(x) \leq i(|x|)$ and $x \in Q$, thus $R_n(x) \in Q$.
  If $x \notin i(n) \dash Q$, then there are two cases.
  In the first, $\kappa(x) > i(|x|)$, in which case $R_n(x) = \bot$, which is not in $Q$ by construction.
  In the second case, $x \notin Q$ so $R_n(x) \notin Q$.

  Finally, we consider the value of $f(\kappa(R_n(x)))$.
  If $\kappa(x) \leq i(n)$, then by construction
  \[
  f(\kappa(R_n(x)) \leq f(\kappa(x)) \leq f(i(n)) \leq \min(n^{O(1)}, O(\log^d n)).
  \]
  On the other hand, if $\kappa(x) > i(n)$, then $\kappa(R_n(x)) = \kappa(\bot) = O(1)$, which is bounded above by both $n^{O(1)}$ and $O(\log^d n)$ for all but finitely many $n$.
  Thus we have shown that $f(\kappa(R_n(x)))$ satisfies the upper bound required by \autoref{lem:reducetonc} and the conclusion, $i(n) \dash Q$ is in $\NC^d$, follows.
\end{proof}

\subsection{Approximable optimization problems}

% % Foreword %
%
% %% Context (anyone - why now?) %%
%
% What is the current situation, and why is the need so important?
%
\autoref{thm:onesdepth} shows an inherently sequential optimization problem with that is in $\para \NC$ when parameterized.
%
% %% Need (readers - why you?) %%
%
% Why is this relevant to the reader, and why does something need to be done?
% (Also reference relevant existing work.)
%
Let us explore the possibility of parameterized parallel algorithms from optimization problems more generally.
(This has been done before only for efficient algorithm for intractable optimization problems.)
%
% %% Task (author - why me?) %%
%
% What was undertaken to address the need?
%
We show how a certain kind of approximation scheme for an optimization problem induces a highly parallel algorithm for the standard parameterized problem derived from the optimization problem.
%
% %% Object (document - why this document?) %%
%
% What does this document cover?
%
This section provides the necessary definitions and generic theorems for this framework.
%

% % Summary %
%
% %% Findings (author - what?)
%
% What did the work reveal when performing the task?
%
We prove that an approximation scheme with appropriate size and depth bounds implies a $\para \NC$ algorithm (\autoref{thm:encasfpp}), and show how this applies to the maximum flow problem under a derandomization assumption (\autoref{thm:maxflow}).
%
% %% Conclusion (readers - so what?)
%
% What did the findings mean for the audience?
%
This means that both existing and newly discovered approximation schemes for optimization problems may present an alternate method of parallelization (via parameterization).
%
% %% Perspective (anyone - what now?)
%
% What should be done next?
One thing we are unable to show in this section is an optimization problem whose budget problem is $\P$-complete and whose standard parameterization is in $\para \NC$ but for which no $\ENCAS$ exists, so we postpone that for future work.

%% Yet another way to do this is to find an optimization problem whose budget problem is $\P$-complete while admitting a highly parallel approximation scheme.

We start with the necessary definitions for optimization problems and approximation schemes.
Some of these definitions appear in \autoref{chp:optimization}, but we repeat them here in a more concise form so that this section is self-contained.

\begin{definition}
  An \emph{optimization problem} $O$ is a four-tuple $(I, S, m, t)$, where $I$ is the set of instances, $S$ is the set of pairs $(x, w)$ where $w$ is a solution for $x$, the function $m$ computes the \emph{measure} (or \emph{objective value}) for such a pair, and $t$ is either $\min$ or $\max$.
\end{definition}

\begin{definition}\label{def:standard}
  The \emph{standard parameterization} of a minimization problem $O$, denoted $p\dash{O}$, is $(Q, \kappa)$, where $Q = \{ (x, k) \, | \, m^*(x) \leq k \}$ and $\kappa(x, k) = k$.
  The inequality is reversed for a maximization problem.
\end{definition}

\begin{definition}
  Suppose $(I, S, m, t)$ is an optimization problem and $(x, y) \in S$.
  The \emph{performance ratio} of the solution $y$ (with respect to $x$), denoted $R(x, y)$, is defined by
  \[
  R(x, y) = \max \left(\frac{m(x, y)}{m^*(x)}, \frac{m^*(x)}{m(x, y)}\right)
  \]
\end{definition}

The performance ratio $R(x, y)$ is a number in the interval $[1, \infty)$.
The closer $R(x, y)$ is to 1, the better the solution $y$ is for $x$, and the closer $R(x, y)$ to $\infty$, the worse the solution.

\begin{definition}
  An \emph{approximation scheme} for an optimization problem is a function $A$ such that for all $x$ and all positive integers $k$ we have $(x, A(x, k)) \in S$ and $R(x, A(x, k)) \leq 1 + \frac{1}{k}$.
\end{definition}

An approximation scheme induces a family of functions, $\{A_k\}_{k \in \mathbb{N}}$, that form progressively better approximations for the optimization problem.
A problem is in $\NCAS$ if it admits an approximation scheme whose slices are in $\NC$.
The problem is in $\FNCAS$ if it admits an approximation scheme that is in $\NC$ with respect to both inputs $n$ and $k$.

%% \begin{definition}
%%   An optimization problem $O$ has an \emph{$\NC$ approximation scheme} if there is an approximation scheme $A$ for $O$ such that for each $k$, we have $A_k \in \FNC$, where $A_k(x) = A(x, k)$ for each $x$.
%% \end{definition}

\begin{definition}
  Suppose $O$ is an optimization problem with $O = (I, S, m, t)$ with $I$ and $S$ in $\NC$ and $m$ in $\FNC$.
  An optimization problem $O$ is in $\NCAS$ if there is an approximation scheme $A$ for $O$ such that for each $k$, we have $A_k \in \FNC$, where $A_k(x) = A(x, k)$ for each $x$.
  The problem is in $\FNCAS$ if there is an approximation scheme $A$ for $O$ such that $A \in \FNC$ (i.e. on both inputs).
\end{definition}

These two complexity classes lead us to a natural interpolation using the ideas of parameterized complexity theory.
This definition is adapted from \autocite[Definition~1.31]{fg06}

\begin{definition}[$\ENCAS$]
  An optimization problem $O$ is in $\ENCAS$ if there is a circuit family $\{A_{n, k}\}$ and a circuit-computable function $f$ such that
  \begin{itemize}
  \item $\{A_{n, k}\}$ is an approximation scheme for $O$,
  \item $\size(A_{n, k}) \leq f(k) n^{O(1)}$,
  \item $\depth(A_{n, k}) \leq f(k) + \log^{O(1)} n$.
  \end{itemize}
\end{definition}

When we consider $k$ as a parameter, then $\ENCAS$ interpolates between $\FNCAS$ and $\NCAS$.
If $f(k)$ is polylogarithmic in $n$, then the definition yields $\FNCAS$.
If $f(k)$ is considered a fixed constant, then the definition yields $\NCAS$.

\begin{proposition}\label{prop:encas}
  $\FNCAS \subseteq \ENCAS \subseteq \NCAS$.
\end{proposition}

%% \todo{Positive linear programming is in $\NCAS$, is it also in $\ENCAS$?}
%% positive linear programming has a depth bound of about f(k) \log^3 n, so no.

We can use an $\ENCAS$ algorithm to construct a $\para \NC$ algorithm for the standard parameterization of an optimization problem.
The converse does not hold: as a counterexample, the minimum vertex cover problem is not in $\ENCAS$ (since no polynomial-time approximation algorithm with approximation ratio better than $7 / 6$ exists \autocite[Theorem~8.1]{hastad01}) but in $\para \NC$ \autocite[Theorem~4.5]{bst15}.
This theorem is an adaptation of \autocite[Theorem~1.32]{fg06}.

\begin{theorem}\label{thm:encasfpp}
  Let $O$ be an optimization problem.
  If $O$ is in $\ENCAS$, then $p\dash{O}$ is in $\para \NC$.
\end{theorem}
\begin{proof}
  Assume without loss of generality that $O$ is a minimization problem; the proof is similar if it is a maximization problem.
  Let $\{m_n\}$ be the $\NC$ circuit family that computes the measure function.
  Let $\{A_{n, k}\}$ be the circuit family such that
  \begin{itemize}
  \item $R(x, A_{n, k}(x, k)) \leq 1 + \frac{1}{k}$ for each $x$ and $k$,
  \item $\size(A_{n, k}) \leq f(k) n^{O(1)}$,
  \item $\depth(A_{n, k}) \leq f(k) + O(\log^{O(1)} n)$,
  \end{itemize}
  for some circuit-computable function $f$.
  Define the circuit family $\{C_{n, k}\}$ as
  \[
  C_{n, k}(x, k) = 1 \iff m(x, A_{n, k + 1}(x, k + 1)) \leq k,
  \]
  so $C_{n, k}$ outputs 1 if and only if the approximate solution corresponding to parameter $k + 1$ measures less than $k + 1$.
  (The function $m$ is really a circuit as well, chosen from a family of circuits depending on the number of bits in its inputs.)

  The size of $C_{n, k}$ is $O(\size(m) + \size(A_{n, k + 1}))$ and the depth is $O(\depth(m) + \depth(A_{n, k + 1})$.
  For some sufficiently large circuit-computable function $f'$, the size and depth bounds are $f'(k + 1) n^{O(1)}$ and $f'(k + 1) + O(\log^{O(1)} n)$, respectively.
  It remains to show correctness of $C_{n, k}$.

  Let $x$ be a string, let $k$ be a natural number, and let $y = A_{n, k + 1}(x, k + 1)$.
  If $C_{n, k} = 1$, then $m(x, y) \leq k$, so $m^*(k) \leq k$ and therefore $(x, k) \in p \dash O$.
  For the converse, if $C_{n, k} = 0$, then $m(x, y) \geq k + 1$, so
  \[
  m^*(x) \geq \frac{m(x, y)}{1 + \frac{1}{k + 1}} \geq \frac{k + 1}{1 + \frac{1}{k + 1}} = \frac{(k + 1)^2}{k + 2} > k.
  \]
  Thus $(x, k) \notin p \dash O$.
  Therefore, we conclude that $p \dash O$ is in $\para \NC$.
\end{proof}

%% \todo{Show an example of an optimization problem whose budget problem is $\P$-complete and whose standard parameterization is in $\para \NC$ but for which no $\ENCAS$ exists.}

Our goal now reduces to finding an optimization problem in $\ENCAS$ whose budget problem is $\P$-complete.
We can provide one under a derandomization assumption.
%% There exist problems that are $\P$-complete and have randomized $\FNCAS$ algorithms, so under a derandomization assumption, we can show a problem that fits the requirement.

\begin{definition}[\textsc{Maximum Flow}]
  \mbox{} \\
  \begin{tabular}{r p{9.2cm}}
    \textbf{Instance:} & directed graph $G$, a natural number capacity $c_e$ for each edge $e$, source node $s$, and target node $t$. \\
    \textbf{Solution:} & flow $F$, defined as a real number $F_e$ for each edge $e$ such that $F_e \leq c_e$ and at each vertex the total in-flow is at least the total out-flow. \\
    \textbf{Measure:} & total in-flow at $t$. \\
    \textbf{Type:} & maximization.
  \end{tabular}
\end{definition}

%% edge flows can have weights exponential in n

\begin{theorem}\label{thm:maxflow}
  If $\NC = \RNC$, then the budget problem for \textsc{Maximum Flow} is $\P$-complete and the standard parameterization is in $\para \NC$.
\end{theorem}
\begin{proof}
  The budget problem for \textsc{Maximum Flow} is $\P$-complete \autocite[Problem~A.4.4]{ghr95}.
  The \textsc{Maximum Flow} problem is in randomized $\FNCAS$ \autocite[Theorem~4.5.2]{dsst97}.
  If $\NC = \RNC$, then randomized $\FNCAS$ equals deterministic $\FNCAS$.
  Thus, the problem is in $\ENCAS$, by \autoref{prop:encas}.
  Finally, the standard parameterization is in $\para \NC$ by \autoref{thm:encasfpp}.
\end{proof}

%% \todo{Can the randomization part of the RNC algorithm for MaxFlow be absorbed in the fixed-parameter part of the algorithm?}

% The positive linear programming problem is only in NCAS, not FNCAS.
