\newenvironment{abstractpage}
  {\thispagestyle{plain}}
  {}

\begin{abstractpage}
  \begin{center}
    \textbf{\uppercase{Parallelism with limited nondeterminism}}\\
    \textbf{\uppercase{Jeffrey Finkelstein}}\\
    Boston University Graduate School of Arts and Sciences, 2017\\
    Major Professor: Steven Homer, Ph.D., Professor of Computer Science
  \end{center}
  \begin{center}
    ABSTRACT
  \end{center}
  % % Foreword %
  %
  % %% Context (anyone - why now?) %%
  %
  % What is the current situation, and why is the need so important?
  %
  Computational complexity theory studies which computational problems can be solved with limited access to resources.
  The past fifty years have seen a focus on the relationship between intractable problems and efficient algorithms.
  %
  % %% Need (readers - why you?) %%
  %
  % Why is this relevant to the reader, and why does something need to be done?
  % (Also reference relevant existing work.)
  %
  However, the relationship between inherently sequential problems and highly parallel algorithms has not been well studied.
  Are there efficient but inherently sequential problems that admit some relaxed form of highly parallel algorithm?
  %
  % %% Task (author - why me?) %%
  %
  % What was undertaken to address the need?
  %
  In this dissertation, we develop the theory of structural complexity around this relationship for three common types of computational problems.
  %
  % %% Object (document - why this document?) %%
  %
  % What does this document cover?
  %

  %
  % % Summary %
  %
  % %% Findings (author - what?)
  %
  % What did the work reveal when performing the task?
  %
  Specifically, we show tradeoffs between time, nondeterminism, and parallelizability.
  By clearly defining the notions and complexity classes that capture our intuition for parallelizable and sequential problems, we create a comprehensive framework for rigorously proving parallelizability and non-parallelizability of computational problems.
  %
  % %% Conclusion (readers - so what?)
  %
  % What did the findings mean for the audience?
  %
  This framework provides the means to prove whether otherwise tractable problems can be effectively parallelized, a need highlighted by the current growth of multiprocessor systems.
  %
  % %% Perspective (anyone - what now?)
  %
  % What should be done next?
  The views adopted by this dissertation---alternate approaches to solving sequential problems using approximation, limited nondeterminism, and parameterization---can be applied practically throughout computer science.
\end{abstractpage}
